\documentclass[12pt,A4]{article} 
    
% formato da página
\usepackage[top=2.5cm,bottom=2.5cm,left=2.5cm,right=2.5cm]{geometry}
    
%% Escrevendo em português
\usepackage[brazil]{babel}
\usepackage[utf8]{inputenc}
\usepackage{hyperref}
%\usepackage{times} 
\usepackage{graphicx}

% Definindo espaçamentos
\setlength{\parindent}{0pc}
\setlength{\parskip}{0pt}
\linespread{1.5} 

\newcommand{\sepitem}{\vspace{0.1in}\item} 
\newcommand{\titulo}{\item \textbf}
\begin {document}
\small{
    \title{
        \vspace{-2cm}
        {\small MAC5786 - Princípios de IHC \hfill DCC/IME/USP}
    \vspace{0.1in}\\
        Como escrever a monografia de MAC5785\\
        {\small coloque o seu título na forma de uma pergunta}
}
\vspace{-0.6in} 
    \author{Seu Nome  \{\textit{seu.email@ime.usp.br}\}
\vspace{-0.6in} 
}
    \date{}
\maketitle
}
\vspace {-0.3in}
\thispagestyle{empty}

% ===================================================================
\section{Resumo}

Esse esqueleto mostra o padrão que você deve utilizar para escrever a sua monografia.
Escreva aqui um breve resumo do seu trabalho, com 10 a 15 linhas.
O resumo deve conter uma breve descrição do tema, e do objetivo da sua
monografia.
Ao colocar o tema na forma de pergunta, você deixará o objetivo da monografia mais claro e, obviamente, você deve explorar as alternativas para responder a pergunta. 

% ===================================================================
\section{Introdução}
\label{sec:Introducao}

A introdução deve descrever o tema, definindo a motivação e justificando a sua importância e relevância. Apresente também os impactos que teríamos ao "responder" a pergunta proposta pelo seu tema. Possivelmente, a partir das
introduções dos demais artigos, você possa costurar uma breve
descrição dos problemas atuais mais relevantes, motivação, contribuições, etc.

% ===================================================================
\section{Estado da arte: discussão crítica dos artigos}
\label{sec:Artigos}

Nessa sessão você deve tentar apresentar os artigos como exemplos de
trabalhos que tentaram responder a pergunta que você colocou. Em particular, discuta a metodologia e contribuições de cada artigo. Junte as descrições caso as metodologias sejam semelhantes. 

Procure apresentar os artigos de forma organizada, por exemplo,
dividindo-os em subseções e assim dar continuidade à leitura. 
Figuras, como a figura~\ref{fig:Cabines}
são também muito importantes para o 
entendimento do texto. Porém, caso você venha a utilizar uma figura de
um dos trabalhos que você leu, forneça os créditos devidamente no
caption da própria figura \footnote{Embora o correto seria obter permissão de
reprodução do próprio autor.}.

\begin{figure} 
    \begin{center}
        \begin{tabular}{cc}
    \includegraphics[width=0.3\textwidth]{cabine0.png} & 
    \includegraphics[width=0.3\textwidth]{cabine1.png} \\
    outra & linha 
    \end{tabular}
    \caption{Essa é uma figura composta. A imagem da cabine foi retirada de \cite{hartley:04}. }
\end{center}
\label{fig:Cabines}
\end{figure} 

% ===================================================================
\subsection{Referências}
\label{sec:Referencias}
Quando necessário, além dos artigos que você apresentar, utilize também
outras referências. Para isso use o comando \texttt{cite} do
latex. Por exemplo, Bradski \cite{bradski:opencv} apresenta um manual do OpenCV e, para saber mais sobre visão e processamento de imagens você pode consultar \cite{hartley:04,gon:dip}. Preferencialmente, aprenda a usar o \href{https://en.wikibooks.org/wiki/LaTeX/Bibliography_Management}{\textbf{bibtex}}.


% ===================================================================
\section{Conclusão}
\label{sec:Discussao}
Na última sessão, você deve resumir o estado da arte e suas limitações, 
apresentando também os desafios que ainda restam e, se você puder, possíveis alternativas para resolver esses desafios.


\begin{thebibliography}{10}
\bibitem{bradski:opencv}
G.~Bradski and A.~Kaehler.
\newblock{Learning OpenCV: Computer Vision with the OpenCV Library}. 
\newblock{O'Reilly}, 2008.
\bibitem{hartley:04}
    R.I.~Hartley and A.~Zisserman.
    \newblock {\em Multiple View Geometry in Computer Vision: 2nd Edition}.
    \newblock Morgan Kaufmann, 2004.
\bibitem{gon:dip}
    R.C.~Gonzalez and R.~E.~Woods.
    \newblock {\em Digital Image Processing}.
    \newblock 2nd edition, Prentice Hall, 2002.
\end{thebibliography}

\vfill

%\raggedleft
%{\sc novembro/2010}

\end{document}

